\section*{3. Предлагаемое решение и его особенности}
\addcontentsline{toc}{section}{3. Предлагаемое решение и его особенности}

На основе выявленных недостатков существующих решений было сформулировано ядро предлагаемого приложения, которое позиционируется как интеллектуальный фитнес-компаньон. Его архитектура строится на трех фундаментальных принципах: гибкость, адаптивность и комплексность.

Во-первых, решена проблема жесткости тренировочных планов. В отличие от распространенной схемы, где пропуск одного занятия нарушает всю программу, каждая тренировка в приложении представляет собой самодостаточный комплекс. Пользователь имеет полный контроль над выполняемым набором упражнений внутри сессии, что позволяет гибко подстраивать нагрузку под текущее состояние, время и цели без чувства «срыва плана».

Во-вторых, реализована система динамической адаптации сложности на основе методологии, описанной в разделе 2.2. Приложение анализирует историю тренировок пользователя по простому правилу:
\begin{itemize}
    \item Если пользователь стабильно выполняет запланированные упражнения, система предлагает увеличить нагрузку.
    \item Если упражнения регулярно остаются невыполненными, система рекомендует уменьшить их количество или продолжительность.
\end{itemize}

В-третьих, приложение интегрирует модуль учета сна. Пользователь вручную вносит данные о продолжительности отдыха, а система использует эту информацию для корректировки рекомендаций. Недостаточный сон служит дополнительным сигналом для снижения тренировочной нагрузки и рекомендации увеличить время отдыха.

Реализованная функциональность включает:
\begin{itemize}
    \item Бэкенд с системой регистрации и аутентификации пользователей.
    \item Каталог готовых тренировок с возможностью гибкой настройки параметров выполнения.
    \item Возможность создания собственных планов тренировок.
    \item Процесс проведения тренировки со встроенным таймером и фоновым аудиосопровождением.
    \item Полную историю выполненных занятий.
    \item Модуль для ручного ввода и статистического анализа продолжительности сна с генерацией советов.
    \item Модуль «Советы по тренировкам», который в реальном времени анализирует историю активности и, при выявлении нескольких незавершенных тренировок подряд, предлагает адаптировать план для обеспечения устойчивого прогресса.
\end{itemize}

Таким образом, разработанное приложение представляет собой целостную экосистему, которая не только предоставляет инструменты для тренировок, но и активно анализирует пользовательское поведение. За счет интеграции модулей адаптации сложности, анализа сна и интеллектуальных советов система формирует замкнутый цикл обратной связи «приложение–пользователь». Это позволяет динамически корректировать нагрузку, предотвращая выгорание и способствуя достижению фитнес-целей с учетом индивидуальных особенностей и текущего состояния пользователя.