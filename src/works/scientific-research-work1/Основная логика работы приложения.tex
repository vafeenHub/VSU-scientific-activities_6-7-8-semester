\section*{5. Основная логика работы приложения}
\addcontentsline{toc}{section}{5. Основная логика работы приложения} % Добавляем в оглавление
Основную логику работы приложения, которая изображена ниже на рис. ~\ref{fig:fig10}, можно разделить на 2 части:
\begin{enumerate}
	\item Просмотр информации о занятиях
	\item Изменение настроек приложения
\end{enumerate}

\begin{figure}[H]
	\centering
	\includegraphics[width=0.8\textwidth]{10.png}
	\caption{Диаграмма вариантов использования приложения}
	\label{fig:fig10}
\end{figure} 

\subsection*{5.1. Логика работы компонента «Главный экран»}
\addcontentsline{toc}{subsection}{5.1. Логика работы компонента «Главный экран»} % Добавляем в оглавление

Логика работы просмотра расписания в приложении реализована через несколько ключевых компонентов: интерфейсы, реализации и клиенты. Функциональность главного экрана можно увидеть на рис. \ref{fig:fig11}.
Эти компоненты взаимодействуют друг с другом для обеспечения функциональности отображения актуального расписания на сегодня, на другие дни, редактирования заметок. При разработке логики получения расписания важно учитывать несколько ключевых моментов, таких как получение данных от удаленного сервера, обработка ответа, управление локальными данными и обеспечение удобного пользовательского интерфейса. В нашем приложении мы используем MVVM архитектуру и DI с помощью Koin для организации кода.

Весь процесс можно разбить на следующие основные компоненты \cite{ref10}: 
\begin{enumerate}
    \item Репозитории данных об учебных занятиях, преподавателях и группах (\texttt{LessonRemoteRepository}, \texttt{GroupRemoteRepository}, \\  \texttt{TeacherRemoteRepository}): эти компоненты отвечают за взаимодействие с данными от удаленного сервера об учебных занятиях, преподавателях и группах. Они осуществляют запросы к удаленному серверу. 
    \item Локальные репозитории данных об учебных занятиях, преподавателях и группах (\texttt{LessonLocalRepository}, \texttt{GroupLocalRepository}, \\  \texttt{TeacherLocalRepository}): эти компоненты отвечают за взаимодействие с локальной базой данных и реализуют операции чтения/записи данных, которые приходят с сервера для кэширования и работы приложения без интернета.
\end{enumerate}

\begin{figure}[H]
    \centering
    \includegraphics[width=0.8\textwidth]{11.png}
    \caption{Диаграмма вариантов использования компонента «Главный экран»}
    \label{fig:fig11}
\end{figure}

\subsection*{5.2. Логика работы компонента «Настройки»}
\addcontentsline{toc}{subsection}{5.2. Логика работы компонента «Настройки»} % Добавляем в оглавление


Компонент настроек (SettingsScreen) отвечает за предоставление пользователю возможности изменять параметры приложения, такие как тема оформления, а также за доступ к некоторым внешним ресурсам, например поддержке.

Весь процесс можно разбить на следующие основные компоненты:
\begin{enumerate}
    \item Компонент «Настройки» (\texttt{SettingsScreen}): фрагмент, отвечающий за отображение и управление настройками приложения. Он содержит интерфейс для изменения темы, отправки отзывов, просмотра пользовательского соглашения и т. д.
    \item \texttt{SettingsScreenViewModel}: ViewModel, который обрабатывает бизнес-логику, связанную с настройками. Он взаимодействует с репозиторием и юзкейсами для получения и сохранения данных.
    \item Менеджер настроек (\texttt{SettingsManager}): интерфейс, определяющий методы для доступа к хранилищу данных настроек. Включает методы для сохранения и получения данных настроек, а также для изменения темы.
    \item Хранилище настроек (\texttt{SharedPreferences}): реализация хранилища настроек, которую использует \texttt{SettingsManager} для сохранения настроек.
\end{enumerate}

\begin{figure}[H]
    \centering
    \includegraphics[width=0.8\textwidth]{12.png}
    \caption{Диаграмма вариантов использования компонента «Настройки»}
    \label{fig:fig12}
\end{figure}

Взаимодействие с компонентом настроек, изображенное на рис. \ref{fig:fig12}, производится в соответствии со следующим планом:
\begin{itemize}
    \item При открытии компонента на экране создается пользовательский интерфейс, и он связывается с ViewModel. Это обеспечивает связь между отображаемыми данными и бизнес-логикой. При загрузке компонента текущие настройки (например, состояние переключателя темы) запрашиваются из ViewModel. ViewModel получает данные \\ от \texttt{SettingsManager}, который взаимодействует с хранилищем. Текущие настройки отображаются в интерфейсе.
    \item Пользователь взаимодействует с элементами интерфейса, такими как переключатель темы, кнопки для отправки отзывов или просмотра кода приложения, а также выбор роли/группы/подгруппы. Когда пользователь изменяет состояние переключателя темы, новое состояние сохраняется через ViewModel и \texttt{SettingsManager}. Тема приложения немедленно обновляется, чтобы отразить новое состояние. При нажатии на соответствующие элементы интерфейса ViewModel вызывает методы интерактора для выполнения этих действий. Юзкейс использует внешний навигатор для отправки письма, открытия ссылки или браузера. После изменения настроек интерфейс обновляется, чтобы отразить текущие параметры. Например, после переключения темы приложение немедленно переключается в дневной или ночной режим.
    \item Все изменения настроек сохраняются в хранилище данных через репозиторий. Это гарантирует, что при следующем запуске приложения будут применены последние сохранённые настройки.
\end{itemize}

\newpage
