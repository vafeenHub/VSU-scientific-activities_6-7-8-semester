% !TEX TS-program = lualatex
\documentclass{vsureport}

\usepackage[dvipsnames]{xcolor}
\usepackage{listings}
\usepackage{enumitem}
\usepackage{fancyhdr}

% настройка нумерации страниц
\pagestyle{fancy}
\fancyhf{}
\fancyhead[C]{\thepage} % Номера на левых и правых страницах
\renewcommand{\headrulewidth}{0pt} % Убираем линию

\begin{document} 

% !TEX TS-program = lualatex
\begin{center} 
\thispagestyle{empty} % выключаем отображение номера для этой
\hfill \break 

\large{МИНОБРНАУКИ РОССИИ}\\ 
\footnotesize{ФЕДЕРАЛЬНОЕ ГОСУДАРСТВЕННОЕ БЮДЖЕТНОЕ}\\  
\footnotesize{ОБРАЗОВАТЕЛЬНОЕ УЧРЕЖДЕНИЕ ВЫСШЕГО ОБРАЗОВАНИЯ}\\ 
\small{«ВОРОНЕЖСКИЙ ГОСУДАРСТВЕННЫЙ УНИВЕРСИТЕТ»}\\ 
\small{(ФГБОУ ВО «ВГУ»)}\\ 
\hfill \break 
\normalsize{Факультет прикладной математики, информатики и механики}\\ 
\hfill \break 
\normalsize{Кафедра математического обеспечения ЭВМ}\\ 
\hfill\break 
\large{Разработка мобильного приложения для отслеживания расписания на платформе Android}\\ 
\hfill \break 
\normalsize{Курсовая работа по дисциплине}\\ 
\normalsize{Б1.О.31 Проектирование информационных систем}\\ 
\hfill \break 
\normalsize{02.03.02 Фундаментальная информатика и информационные технологии}\\ 
\hfill \break 
\normalsize{Инженерия программного обеспечения}\\ 
\hfill \break 
\end{center} 

\normalsize{  
\begin{tabular}{l@{\hspace{4cm}}r}
Зав. кафедрой & \underline{\hspace{3cm}} \\
д.т.н., профессор Абрамов Г.В. & \underline{\hspace{3cm}} \\[1cm]
Обучающийся & \underline{\hspace{3cm}} \\
Вафин А.Р. & \underline{\hspace{3cm}} \\[1cm]
Руководитель & \underline{\hspace{3cm}} \\
к.ф.-м.н., доцент Болотова С.Ю. & \underline{\hspace{3cm}} \\
\end{tabular} 
}\\
\hfill\break

\begin{center} Воронеж 2024 \end{center}
% КОНЕЦ ТИТУЛЬНОГО ЛИСТА

\newpage

\thispagestyle{empty} % без номера страницы
\tableofcontents
\newpage

% !TEX TS-program = lualatex
\section*{Введение}
\addcontentsline{toc}{section}{Введение} % Добавляем в оглавление

В современном мире мобильные приложения стали неотъемлемой частью повседневной жизни, охватывая все новые аспекты человеческой деятельности. Особую значимость приобретает их использование в сфере здоровья и физической культуры, где они трансформируются из простых инструментов отслеживания в полноценных цифровых персональных тренеров и мотиваторов. Разработка специализированных фитнес-приложений для платформы Android представляет собой актуальную и востребованную задачу, обусловленную глобальным трендом на здоровый образ жизни, повышением внимания к физическому благополучию и повсеместным использованием смартфонов.

Актуальность разработки многофункционального фитнес-приложения для Android обусловлена комплексом ключевых факторов. Доминирующее положение операционной системы Android на глобальном рынке обеспечивает потенциально массовый охват пользовательской аудитории, от начинающих энтузиастов до опытных атлетов. Также современные технологии разработки позволяют создавать интуитивно понятные и функциональные интерфейсы для эффективной работы с обширными каталогами тренировок и упражнений, что обеспечивает пользователям легкий доступ к структурированной библиотеке тренировочных методик.

Таким образом, создание интеллектуального мобильного фитнес-приложения для Android представляет собой прикладную задачу, находящуюся на стыке разработки программного обеспечения, спортивной медицины, диетологии и поведенческой психологии. Его успешная реализация способна не только коммерциализироваться на массовом рынке, но и внести существенный вклад в улучшение здоровья и качества жизни населения.
\newpage

\section*{1. Постановка цели и задач}
\addcontentsline{toc}{section}{1. Постановка цели и задач}

Целью данной работы является разработка мобильного фитнес-приложения для платформы Android, предоставляющего пользователям структурированный каталог тренировок и инструменты для создания индивидуальных планов занятий. Для достижения этой цели необходимо решить следующие задачи:

\begin{enumerate}
	\item Проектирование и реализация пользовательского интерфейса приложения, включая создание интуитивно понятной навигации между экранами, разработку системы отображения тренировок и упражнений, а также реализацию инструментов для составления персональных планов занятий.
	
	\item Разработка модуля работы с данными, включая проектирование и реализацию локальной базы данных для хранения информации о тренировках, упражнениях и пользовательских настройках, а также создание системы управления этими данными.
	
	\item Реализация бизнес-логики приложения, включая механизмы фильтрации и поиска тренировок, систему составления индивидуальных планов занятий на основе выбранных параметров (уровень подготовки, целевые группы мышц, доступное оборудование), а также функционал отслеживания прогресса.
	
	\item Тестирование работоспособности приложения, включая проверку корректности отображения интерфейса, тестирование функциональности всех компонентов системы, а также валидацию работы алгоритмов составления тренировочных планов.
\end{enumerate}

\newpage

\subsection*{1.1. Обзор литературных источников и технологий}
\addcontentsline{toc}{subsection}{1.1. Обзор литературных источников и технологий}

Разработка современного фитнес-приложения требует опоры на актуальные технологические решения и понимание принципов организации тренировочного процесса. В рамках исследования были изучены как технические публикации, так и материалы, касающиеся методологии тренировок.

В области технологий стека Android ключевыми источниками выступила официальная документация Google для разработчиков \cite{ref5, ref6}. Особое внимание было уделено изучению возможностей фреймворка Jetpack Compose, который представляет собой парадигмальный сдвиг в создании пользовательских интерфейсов, предлагая декларативный подход вместо императивного \cite{ref5}. Анализ документации к библиотекам Room \cite{ref3} и Retrofit подтвердил их статус как стандартных решений для работы с локальным хранением данных и сетевыми запросами соответственно, обеспечивающих безопасность типов и сокращение шаблонного кода.

С методологической точки зрения, при проектировании логики приложения учитывались принципы прогрессивной перегрузки и необходимости адекватного восстановления \cite{ref2}. Идея адаптации тренировочного плана на основе обратной связи от пользователя, а не жесткого следования предустановленному графику, стала центральной. Исследования, касающиеся влияния качества и количества сна на когнитивные функции и физическое восстановление \cite{ref7}, обосновали необходимость включения в приложение не просто трекера сна, но и системы аналитики, выдающей содержательные рекомендации. Таким образом, техническая реализация опирается на проверенные отраслевые стандарты, а функциональное наполнение — на актуальные представления о эффективном и безопасном тренировочном процессе.
\newpage

\section*{2. Предлагаемое решение и его особенности}
\addcontentsline{toc}{section}{2. Предлагаемое решение и его особенности}

На основе выявленных недостатков существующих решений было сформулировано ядро предлагаемого приложения, которое позиционируется как интеллектуальный фитнес-компаньон. Его архитектура строится на трех фундаментальных принципах: гибкость, адаптивность и комплексность.

Во-первых, решена проблема жесткости тренировочных планов. В отличие от распространенной схемы, где пропуск одного занятия нарушает всю программу, каждая тренировка в приложении представляет собой самодостаточный комплекс. Пользователь имеет полный контроль над выполняемым набором упражнений внутри сессии, что позволяет гибко подстраивать нагрузку под текущее состояние, время и цели без чувства «сорыва плана».

Во-вторых, реализована система динамической адаптации сложности. Приложение анализирует историю тренировок пользователя. Алгоритм отслеживает как последовательные успешные завершения (предлагая увеличение нагрузки), так и рассогласование между средней и плановой продолжительностью занятий, рекомендуя в последнем случае скорректировать структуру тренировки для предотвращения выгорания. Эта функция закладывает основу для персонального подхода.

В-третьих, приложение выходит за рамки простого трекера упражнений, интегрируя модуль учета и анализа сна. Пользователь вручную вносит данные о продолжительности отдыха, а система, аккумулируя статистику, предоставляет не только графики, но и содержательные текстовые рекомендации, предупреждая о рисках, связанных с хроническим недосыпом или избыточным сном.

Реализованная функциональность включает:
\begin{itemize}
    \item бэкенд с системой регистрации и аутентификации пользователей;
    \item каталог готовых тренировок с возможностью гибкой настройки параметров выполнения;
    \item возможность добавления собственных планов тренировок;
    \item процесс проведения тренировки со встроенным таймером и фоновым аудиосопровождением;
    \item полную историю выполненных занятий;
    \item модуль для ручного ввода и статистического анализа продолжительности сна с генерацией советов;
    \item модуль «Советы по тренировкам», который в режиме реального времени анализирует историю активности пользователя: если система фиксирует несколько незавершенных тренировок подряд, она предлагает адаптировать план — например, уменьшить количество упражнений или их продолжительность, обеспечивая тем самым персонализированный и устойчивый прогресс.
\end{itemize}

Таким образом, разработанное приложение представляет собой целостную экосистему, которая не только предоставляет инструменты для тренировок, но и активно анализирует пользовательское поведение. За счет интеграции модулей адаптации сложности, анализа сна и интеллектуальных советов по тренировкам, система формирует замкнутый цикл обратной связи «приложение–пользователь». Это позволяет динамически корректировать нагрузку, предотвращая выгорание и способствуя достижению фитнес-целей с учетом индивидуальных особенностей и текущего состояния пользователя.

\newpage

\section*{3. Анализ средств реализации}
\addcontentsline{toc}{section}{3. Анализ средств реализации} % Добавляем в оглавление

Для достижения поставленных целей и решения задач будут использованы следующие методы анализа и исследования:
\begin{enumerate}
\item Анализ существующих приложений для отслеживания расписания и API.
\item Проектирование пользовательского интерфейса с использованием методов UX/UI дизайна.
\item Разработка и тестирование программного кода приложения.
\end{enumerate}



\subsection*{3.1. Обзор существующих языков программирования}
\addcontentsline{toc}{subsection}{3.1. Обзор существующих языков программирования} % Добавляем в оглавление

При сравнении Java и Kotlin выявлены существенные различия в подходах к разработке. Java, будучи зрелым языком, требует больше ручного управления кодом, включая написание геттеров, сеттеров и других шаблонных конструкций, что увеличивает объем работы и вероятность ошибок. В то же время Kotlin предлагает более высокоуровневый подход с автоматической генерацией boilerplate-кода, например, в data-классах, что делает разработку быстрее и удобнее.
В плане производительности оба языка компилируются в байт-код JVM, однако Kotlin предоставляет встроенные оптимизации, такие как null-безопасность и корутины, которые упрощают асинхронное программирование и снижают риск ошибок. Java, хотя и остается высокопроизводительным, требует ручной оптимизации, особенно при работе с памятью и многопоточностью.
Гибкость Kotlin проявляется в поддержке функционального программирования, включая лямбда-выражения и расширения функций, что позволяет писать более лаконичный и модульный код. Java, несмотря на добавление Stream API и лямбд, все еще уступает в этом отношении, требуя большего количества шаблонного кода.
Поддержка и сообщество Java остаются одними из самых больших, что делает язык надежным выбором для legacy-проектов. Однако Kotlin активно развивается, особенно в Android-разработке, и полностью совместим с Java, что позволяет использовать существующие библиотеки и постепенно переходить на более современный стек.
Тестирование в Kotlin упрощается благодаря лаконичному синтаксису и встроенным возможностям, таким как корутины, тогда как в Java для сложных сценариев часто требуются дополнительные библиотеки. 
В итоге Kotlin был выбран как более современный, безопасный и удобный язык, сочетающий производительность Java с улучшенным синтаксисом и инструментами.


\subsection*{3.2. Обзор существующих фреймворков для интерфейса}
\addcontentsline{toc}{subsection}{3.2. Обзор существующих фреймворков для интерфейса} % Добавляем в оглавление

При сравнении подходов к созданию пользовательского интерфейса явно прослеживаются ключевые различия между традиционной XML-версткой и современным Jetpack Compose. XML-верстка, несмотря на свою зрелость и обширную базу существующих решений, демонстрирует ряд существенных ограничений, таких как жесткое разделение логики и интерфейса, необходимость ручного управления View-элементами и сложности с динамическим обновлением контента. В противоположность этому, Jetpack Compose предлагает единый программный подход на Kotlin, где интерфейс описывается декларативно непосредственно в коде, что значительно упрощает разработку и поддержку.
С точки зрения производительности Compose обладает явными преимуществами благодаря своей системе "рекомпозиции", которая интеллектуально обновляет только измененные элементы интерфейса. В отличие от традиционной View-системы с ее глубокими иерархиями и ручной оптимизацией, Compose обеспечивает более эффективное использование ресурсов устройства, что особенно важно для современных требовательных приложений.
Гибкость Compose проявляется в простоте создания сложных анимированных интерфейсов и кастомизации компонентов. В то время как XML-верстка требует написания значительного объема кода для реализации нестандартных решений, Compose предоставляет встроенные инструменты для анимаций и позволяет легко создавать переиспользуемые компоненты.
Несмотря на то, что XML-верстка имеет более обширное сообщество и документацию из-за своей долгой истории, Jetpack Compose активно развивается при поддержке Google и постепенно становится новым стандартом в Android-разработке. Его возможность сосуществования с традиционными View позволяет осуществлять плавный переход на новые технологии.
В области тестирования Compose также предлагает более удобные решения со встроенными инструментами тестирования, что упрощает процесс проверки UI по сравнению с необходимостью использования дополнительных библиотек в случае XML-верстки.
Таким образом, Jetpack Compose был выбран как современный, производительный и гибкий фреймворк, который значительно ускоряет разработку интерфейсов за счет сокращения шаблонного кода и предоставления мощных инструментов для создания отзывчивых и интерактивных пользовательских интерфейсов.

\subsection*{3.3. Обзор существующих фреймворков базы данных}
\addcontentsline{toc}{subsection}{3.3. Обзор существующих фреймворков базы данных} % Добавляем в оглавление

При выборе подхода к работе с базами данных в Android-приложениях рассматривались два основных решения: нативный SQLite и библиотека Room. Нативный SQLite, являясь стандартным решением для Android, требует значительных усилий при реализации - разработчику необходимо вручную создавать таблицы, писать SQL-запросы и управлять соединениями, что приводит к большому объему шаблонного кода. В отличие от этого, Room предоставляет удобную абстракцию над SQLite, автоматически генерируя необходимый код на основе аннотаций, что значительно сокращает время разработки и уменьшает вероятность ошибок.
С точки зрения производительности оба подхода демонстрируют сопоставимые результаты, так как Room в конечном итоге использует тот же SQLite. Однако Room предлагает дополнительные оптимизации, такие как кэширование запросов и проверка их корректности на этапе компиляции, что позволяет избежать распространенных ошибок, характерных для ручного написания SQL-запросов. При этом нативный SQLite сохраняет преимущество при работе с особо сложными запросами, где может потребоваться тонкая ручная оптимизация.
Гибкость нативного SQLite проявляется в возможности выполнения произвольных запросов и полном контроле над структурой базы данных, однако это требует от разработчика глубоких знаний SQL и тщательного управления миграциями. Room, хотя и ограничивает некоторые низкоуровневые возможности, предоставляет удобные механизмы для работы с транзакциями и миграциями данных, существенно упрощая эти процессы. Особенно ценным является встроенная в Room поддержка LiveData и Flow, позволяющая легко реализовать реактивное обновление интерфейса при изменениях в базе данных.
Несмотря на то, что нативный SQLite имеет более обширную базу знаний и примеров благодаря своей долгой истории, Room активно развивается как часть Android Jetpack и становится стандартом для новых проектов. Его популярность постоянно растет, а сообщество разработчиков расширяется, что обеспечивает хорошую поддержку и наличие актуальных решений для типовых задач.
Важным преимуществом Room являются встроенные инструменты тестирования, такие как возможность создания временных баз данных в памяти, что существенно упрощает процесс написания и выполнения unit-тестов по сравнению с нативным SQLite, где требуется дополнительная настройка тестового окружения.
Таким образом, Room был выбран в качестве основного инструмента для работы с базами данных благодаря своей простоте использования, безопасности типов, отличной интеграции с другими компонентами Android Jetpack и возможности сосредоточиться на бизнес-логике приложения, а не на низкоуровневых деталях работы с SQLite. \cite{ref3}


\subsection*{3.4. Обзор существующих фреймворков для работы с сетью}
\addcontentsline{toc}{subsection}{3.4. Обзор существующих фреймворков для работы с сетью} % Добавляем в оглавление

При разработке сетевого слоя приложения рассматривались два основных подхода: низкоуровневый OkHttp и высокоуровневый Retrofit. OkHttp предоставляет полный контроль над HTTP-запросами, позволяя тонко настраивать все параметры соединения, однако требует значительного объема ручного кода для реализации типовых сценариев. В отличие от него, Retrofit предлагает декларативный подход через аннотированные интерфейсы, автоматизируя создание клиентов и обработку ответов, что существенно сокращает объем шаблонного кода.
С точки зрения производительности оба решения демонстрируют отличные результаты, поскольку Retrofit использует OkHttp в качестве транспортного уровня. OkHttp обеспечивает продвинутые оптимизации, включая поддержку HTTP/2, кэширование и сжатие данных, в то время как Retrofit добавляет удобную абстракцию поверх этих возможностей без потери эффективности.
Гибкость OkHttp проявляется в возможности создания кастомных интерцепторов и полного контроля над запросами, что критично для сложных интеграций. Retrofit, хотя и ограничивает некоторые низкоуровневые возможности, сохраняет доступ к базовым настройкам OkHttp и предоставляет удобные механизмы для адаптации ответов и обработки ошибок.
Оба решения имеют отличную поддержку и активное сообщество благодаря принадлежности к экосистеме Square. OkHttp, как фундаментальная библиотека, используется в большинстве Android-приложений, тогда как Retrofit стал стандартом де-факто для работы с REST API благодаря своей простоте и элегантности API.
В области тестирования Retrofit предлагает более удобный подход, позволяя легко создавать mock-серверы и тестировать API-контракты, в то время как тестирование чистого OkHttp требует больше усилий по настройке тестового окружения.
Таким образом, Retrofit был выбран в качестве основного инструмента для сетевых запросов благодаря простоте реализации типовых сценариев через аннотированные интерфейсы, автоматической сериализации/десериализации данных, гибкой системе адаптеров и конвертеров, полной совместимости с OkHttp, удобным механизмам обработки ошибок и простому процессу тестирования. При этом сохраняется возможность тонкой настройки через кастомные OkHttp-клиенты и интерцепторы, когда это необходимо для сложных случаев интеграции.
\newpage

\input{4. Выбор средств реализации.tex}
\newpage

\section*{4. Реализация}
\addcontentsline{toc}{section}{4. Реализация} % Добавляем в оглавление

{\color{red} Здесь будут этапы реализации в виде схемы} 
% Этапы реализации могут быть представлены в виде схемы (рис.~\ref{fig:fig1}).

% \begin{figure}[htbp]
% 	\centering
% 	\includegraphics[width=1.0\textwidth]{images/1.png}
% 	\caption{Схема этапов реализации приложения}
% 	\label{fig:fig1}
% \end{figure}

\subsection*{4.1. Подключение зависимостей}
\addcontentsline{toc}{subsection}{4.1. Подключение зависимостей} % Добавляем в оглавление

Для написания приложения нужно добавить необходимые библиотеки в Gradle файле на уровне приложения: 
\begin{enumerate}
	\item Retrofit – это библиотека для работы с HTTP-запросами в Android-приложениях. Она используется для взаимодействия с собственным API, обеспечивая удобный и эффективный способ отправки запросов и обработки ответов. Конвертер Gson позволяет автоматически преобразовывать JSON-ответы от API в объекты Kotlin, что упрощает работу с данными.   \cite{ref7}\cite{ref8}
	
	\item Glide – это библиотека для загрузки и отображения изображений. Она используется для отображения картинок котиков на интерфейсе, а также их кэширования для повышения производительности. \cite{ref7}
	
	\item AndroidX Libraries предоставляет современные компоненты и инструменты для разработки Android-приложений, улучшая совместимость и функциональность. \cite{ref7}
	
	\item Jetpack Compose для декларативной верстки экранов приложения. \cite{ref7}
	
	\item Jetpack Navigation: обеспечивает удобный способ навигации между экранами приложения, упрощая управление компонентами и навигационными действиями. 
	
	\item Koin – это библиотека для внедрения зависимостей, которая упрощает создание и управление зависимостями в приложении, улучшая модульность и тестируемость кода.
	
	\item Room – это библиотека для работы с базой данных SQLite, предоставляющая удобный и безопасный способ хранения и управления данными в приложении. Благодаря ей будет происходить хранение информации о треках, плейлистах и избранном.
\end{enumerate}

\subsection*{4.2. Верстка экранов}
\addcontentsline{toc}{subsection}{4.2. Верстка экранов} % Добавляем в оглавление

% Все экраны приложения верстаются с помощью Kotlin Declarative подхода Jetpack Compose. На этом этапе вынесены отдельно строковые и цветовые ресурсы, необходимые шрифты и векторные изображения для иконок. Для всех частей интерфейса используются стандартные UI-элементы Android SDK.[2]
% Приложение состоит из двух основных экранов: расписание и настройки. Смена состояния основного Activity осуществляется при помощи помещения одного из двух фрагментов в Scaffold. Расписание занятий отображается с помощью Column, где каждый элемент списка – это контейнер для события, куда записывается информация о занятии. Пока не начата работа с сетью, оставлены изображения-заглушки и тестовый текст, чтобы убедиться в корректности отображения интерфейса. \cite{ref8}
% Для похожих элементов UI необходимо вынести одинаковые атрибуты в отдельный стиль для дальнейшего повторного использования. Также необходимо добавить поддержку светлой и темной темы, чтобы пользователи могли выбирать наиболее удобный для себя вариант отображения интерфейса. Ниже на рис. ~\ref{fig:main_screen} и рис. ~\ref{fig:settings_screen} изображена итоговая верстка экранов с тестовыми данными.

{\color{red}Здесь будет описание всех экранов с фотками верстки}


\subsection*{4.3. Реализация навигации}
\addcontentsline{toc}{subsection}{4.3. Реализация навигации} % Добавляем в оглавление
На данном этапе необходимо связать сверстанные экраны в единое приложение. Для реализации логики Bottom Bar будет использоваться Jetpack Navigation. При нажатии на элементы расписания необходимо обеспечить переход на соответствующие детальные экраны или выполнить другие действия, такие как открытие дополнительной информации о занятии. \cite{ref8}

{\color{red} Здесь будут скрины экранов}
% \begin{figure}[H]
%     \centering
%     \begin{minipage}{0.48\textwidth}
%         \centering
%         \includegraphics[width=\linewidth]{images/13.png}
%         \caption{Макет главного экрана}
%         \label{fig:main_screen}
%     \end{minipage}
%     \hfill
%     \begin{minipage}{0.48\textwidth}
%         \centering
%         \includegraphics[width=\linewidth]{images/14.png}
%         \caption{Макет экрана настроек}
%         \label{fig:settings_screen}
%     \end{minipage}
% \end{figure}

% На экране настроек при нажатии кнопки «Назад» необходимо вернуться на предыдущий экран. Это реализовано с помощью стандартных механизмов навигации в Android, например, NavController. \cite{ref8}
% До реализации взаимодействия с базой данных полная функциональность некоторых экранов будет ограничена. Однако, основная логика навигации и переходов между экранами будет реализована для обеспечения плавного пользовательского опыта.


\subsection*{4.4. Проверка HTTP-запросов}
\addcontentsline{toc}{subsection}{4.4. Проверка HTTP-запросов} % Добавляем в оглавление

Ниже [на рисунках]
% , на рис. 
% ~\ref{fig:fig2}
% , рис. 
% ~\ref{fig:fig3}
% , рис. 
% ~\ref{fig:fig4}
% , рис. ~\ref{fig:fig5}
, изображены тестовые обращения для проверки запросов к собственному API через Postman. Все данные, необходимые для отображения пользователю, существуют и корректны.  \cite{ref9}



{\color{red} Здесь будут скриншоты из postman}
% \begin{figure}[H]
% 	\centering
% 	\includegraphics[width=0.8\textwidth]{images/2.png}
% 	\caption{Запрос получения преподавателей}
% 	\label{fig:fig2}
% \end{figure}

% \begin{figure}[H]
% 	\centering
% 	\includegraphics[width=0.8\textwidth]{images/3.png}
% 	\caption{Запрос получения групп}
% 	\label{fig:fig3}
% \end{figure}

% \begin{figure}[H]
% 	\centering
% 	\includegraphics[width=0.8\textwidth]{images/4.png}
% 	\caption{Запрос получения расписания конкретной группы}
% 	\label{fig:fig4}
% \end{figure}

% \begin{figure}[H]
% 	\centering
% 	\includegraphics[width=0.8\textwidth]{images/5.png}
% 	\caption{Запрос получения расписания конкретного преподавателя}
% 	\label{fig:fig5}
% \end{figure} 

\subsection*{4.5. Модели данных}
\addcontentsline{toc}{subsection}{4.5. Модели данных} % Добавляем в оглавление

Следующие модели данных, изображенные на рис. {номера рисунков}
% ~\ref{fig:fig6}
, используются для конвертации в них HTTP-ответов от сервера в формат JSON и обработки для дальнейшего помещения в базу данных и отображения на интерфейсе. 

{\color{red}Здесь будут диаграммы моделей данных}
% \begin{figure}[H]
% 	\centering
% 	\includegraphics[width=0.7\textwidth]{images/6.png}
% 	\caption{Диаграмма классов для моделей данных сетевых репозиториев}
% 	\label{fig:fig6}
% \end{figure} 
% \begin{enumerate}
% 	\item LessonDto: модель для представления информации о занятии.
%  	\item TeacherDto: модель для представления информации о преподавателе.
% 	\item GroupDto: модель для представления информации о группе студентов.
% \end{enumerate}
\subsection*{4.6. Логика работы с сетью}
\addcontentsline{toc}{subsection}{4.6. Логика работы с сетью} % Добавляем в оглавление

Основная цель работы с сетью — обеспечение взаимодействия приложения с API, для получения информации о расписании по запросу пользователя. \cite{ref6}

Для работы с сетью в приложении используются следующие ключевые компоненты, диаграмма которых изображена на рис. 
% \ref{fig:fig7}
: 

{\color{red}Здесь будет диаграмма репозиториев для работы с сетью}


% \begin{enumerate}
%     \item Интерфейсы LessonRemoteRepository, GroupRemoteRepository, Teacher-RemoteRepository. Определяют методы, которые принимает объект запроса и возвращает объект ответа. Эти интерфейсы позволяет абстрагироваться от конкретных деталей реализации сетевого взаимодействия, обеспечивая гибкость и возможность замены реализации при необходимости.
	
%     \item Классы LessonRemoteRepositoryImpl, GroupRemoteRepositoryImpl, TeacherRemoteRepositoryImpl. Реализуют интерфейсы Lesson-RemoteRepository, GroupRemoteRepository, TeacherRemoteRepository и используют библиотеку Retrofit для выполнения сетевых запросов. Здесь проверяется сетевое подключение – прежде чем отправить запрос, проверяется наличие активного интернет-соединения. Если соединение отсутствует, возвращается ответ с соответствующим кодом ошибки. А также обрабатывается сам запрос: методы принимают параметры запроса и выполняют сетевой запрос к API. В случае успешного выполнения запроса возвращается ответ с кодом 200 и полученными данными. В случае ошибки (например, при отсутствии соединения или неверном типе запроса) возвращаются соответствующие коды ошибок.\cite{ref9}
% \end{enumerate}

% \begin{figure}[H]
% 	\centering
% 	\includegraphics[width=0.8\textwidth]{images/7.png}
% 	\caption{Диаграмма классов для сетевых репозиториев}
% 	\label{fig:fig7}
% \end{figure}

\subsection*{4.7. Логика работы с базой данных}
\addcontentsline{toc}{subsection}{4.7. Логика работы с базой данных} % Добавляем в оглавление

Для обеспечения надежного и эффективного хранения данных о занятиях в приложении используется библиотека Room, которая предоставляет удобный интерфейс для работы с SQLite базой данных в Android-приложениях. Для реализации функциональности работы с базой данных в приложении используются следующие основные компоненты:

{\color{red}Здесь будет диаграмма классов работы с базой данных, диаграмма схемы базы данных }
% \begin{enumerate}
% 	\item Entity-классы представляют собой модели данных, которые будут сохраняться в базе данных. В приложении используются следующие Entity-классы, которые изображены ниже на рис. ~\ref{fig:fig8}:
% 		\begin{itemize}
% 		\item LessonEntity: представляет информацию о занятии, включая его идентификатор, название, время начала и конца, аудиторию, преподавателя и т. д.
% 		\item TeacherEntity: представляет информацию о преподавателе, включая его идентификатор, имя, должность и т. д.
% 		\item GroupEntity: представляет информацию о группе студентов, включая ее идентификатор, название и т. д.
% 		\item ReminderEntity: представляет информацию о напоминании о занятии
% 		\end{itemize}
% 		\begin{figure}[H]
% 			\centering
% 			\includegraphics[width=0.8\textwidth]{images/8.png}
% 			\caption{Схема базы данных}
% 			\label{fig:fig8}
% 		\end{figure} 
		
% 	\item DAO (Data Access Object) интерфейсы определяют методы для взаимодействия с базой данных. В нашем приложении используются три DAO-интерфейса, которые изображены на рис ~\ref{fig:fig9}:
% 		\begin{itemize}
% 		\item LessonDao: определяет методы для вставки, удаления и получения занятий из базы данных.
% 		\item TeacherDao: определяет методы для вставки, удаления и получения преподавателей из базы данных.
% 		\item GroupDao: определяет методы для вставки, удаления и получения групп из базы данных.
% 		\end{itemize}
% \end{enumerate}



% Эти интерфейсы аннотированы с использованием аннотаций Room, таких как \texttt{@Insert}, \texttt{@Delete} и \texttt{@Query}, которые определяют SQL-запросы, выполняемые для соответствующих операций. \cite{ref10}

% \begin{figure}[H]
% 	\centering
% 	\includegraphics[width=0.8\textwidth]{images/9.png}
% 	\caption{Диаграмма классов базы данных}
% 	\label{fig:fig9}
% \end{figure} 


% \begin{enumerate}[start=3]
%     \item Абстрактный класс базы данных наследуется от \texttt{RoomDatabase} и представляет собой точку входа для взаимодействия с базой данных. В нашем приложении определены следующие классы:
    
%     \begin{itemize}
%         \item AppDatabase: абстрактный класс для работы с таблицей занятий. Содержит абстрактные свойства \texttt{lessonDao}, \texttt{groupDao}, \texttt{teacherDao}, возвращающие экземпляры интерфейсов \texttt{LessonDao}, \texttt{GroupDao} и \texttt{TeacherDao} соответственно.
%     \end{itemize}

%     \item Клиент базы данных предоставляет методы для взаимодействия с базой данных через DAO-интерфейсы. В нашем приложении используется класс \texttt{AppDatabase\_Impl}, который реализует интерфейс \texttt{AppDatabase}. Этот класс определяет методы для сохранения, удаления и получения занятий. Класс \texttt{AppDatabase\_Impl} инкапсулирует логику работы с базой данных, обеспечивая удобный интерфейс для других компонентов приложения. Он использует инъекцию зависимости для получения экземпляра базы данных и обеспечивает выполнение CRUD-операций через соответствующие DAO-интерфейсы.
% \end{enumerate}

\subsection*{4.8. Архитектура приложения}
\addcontentsline{toc}{subsection}{4.8. Архитектура приложения} % Добавляем в оглавление

Приложение будет разработано с соблюдением принципов SOLID и чистой архитектуры. Применение этих принципов обеспечивает высокую степень модульности, гибкости и тестируемости кода, а также упрощает поддержку и расширение функциональности. \cite{ref3}

Чистая архитектура разделяет систему на несколько слоев, каждый из которых имеет чётко определённые задачи и зависимости. Это позволяет легко изменять и масштабировать приложение, а также улучшает его тестируемость. В соответствии с чистой архитектурой в приложении выделены следующие основные слои:

\begin{enumerate}
    \item \textbf{Внешний слой (UI Layer)}: включает все компоненты, связанные с пользовательским интерфейсом, такие как фрагменты и активности. Основная задача UI Layer - отображение данных и взаимодействие с пользователем. Этот слой получает данные из ViewModel и обновляет интерфейс в соответствии с изменениями в данных~\cite{ref3}.
    \item \textbf{Слой данных (Data Layer)}: содержит компоненты, связанные с управлением данными, включая DAO-интерфейсы для работы с Room, реализации сетевых запросов с использованием Retrofit, а также клиентов базы данных и сетевых клиентов. Data Layer отвечает за получение и хранение данных, предоставляя их в Domain Layer через репозитории~\cite{ref3}.
    \item \textbf{Слой домена (Domain Layer)}: содержит бизнес-логику и бизнес-модели. Domain Layer не зависит от других слоев и содержит основные бизнес-правила и интерфейсы репозиториев~\cite{ref3}.
    \item \textbf{Слой приложений (Application Layer)}: содержит ViewModel, которые объединяют бизнес-логику с логикой представления и управления состоянием. ViewModel взаимодействуют с репозиториями для получения данных и обработки пользовательских действий.
\end{enumerate}

Для разделения представления и логики управления данными в приложении используется паттерн MVVM (Model-View-ViewModel). Этот паттерн обеспечивает чёткое разделение обязанностей между компонентами приложения:

\begin{enumerate}
    \item \textbf{Model}: включает бизнес-логику и данные приложения. Модель получает данные из Data Layer и предоставляет их ViewModel.
    \item \textbf{View}: состоит из пользовательского интерфейса и отвечает за отображение данных. View наблюдает за изменениями в ViewModel и обновляет интерфейс в соответствии с этими изменениями.
    \item \textbf{ViewModel}: посредник между View и Model. ViewModel запрашивает данные у Model и предоставляет их View, а также обрабатывает пользовательские действия и обновляет Model.
\end{enumerate}

Для управления зависимостями и их внедрения в приложении используется библиотека Koin. В модулях Koin определяются зависимости и способы их создания. Например, можно определить зависимости для сетевого клиента, базы данных и ViewModel. Это позволяет централизованно управлять зависимостями и облегчает их модификацию. Koin автоматически предоставляет экземпляры классов, когда они требуются, что упрощает тестирование и модульность. Например, ViewModel могут получать необходимые репозитории через Koin, что устраняет жёсткие зависимости\cite{ref10}

\newpage

\section*{Заключение}
\addcontentsline{toc}{section}{Заключение} % Добавляем в оглавление

В ходе работы было разработано мобильное фитнес-приложение для платформы Android, реализующее комплексный подход к организации физических нагрузок и отслеживанию показателей здоровья. Приложение сочетает в себе основные преимущества существующих решений и дополняет их уникальным функционалом, основанным на принципах гибкости, адаптивности и персонализации.

Ключевыми достижениями работы являются:
\begin{itemize}
    \item Интеграция готовой системы тренировок из интернет-ресурсов, где каждая сессия представляет собой самодостаточный комплекс, позволяющий пользователю гибко выбирать упражнения без риска нарушения общей программы;
    \item Реализация интеллектуального модуля адаптации сложности, анализирующего историю занятий и предлагающего корректировки нагрузки на основе устойчивости выполнения плана;
    \item Интеграция модуля учёта сна с системой аналитики и генерации персонализированных рекомендаций, что обеспечивает комплексный подход к восстановлению;
    \item Разработка системы «Советы по тренировкам», которая отслеживает процент завершённых упражнений и предлагает оптимизацию плана для предотвращения выгорания.
\end{itemize}

Все поставленные задачи были успешно выполнены: спроектирован и реализован пользовательский интерфейс с помощью Jetpack Compose, разработана локальная база данных на основе Room, реализована бизнес-логика приложения, включая механизмы фильтрации тренировок и отслеживания прогресса, а также проведено тестирование работоспособности системы.

Разработанное приложение демонстрирует потенциал для дальнейшего масштабирования. В перспективе возможно добавление социальных функций (соревнований, общих челленджей), интеграция с носимой электроникой для автоматического сбора данных о сне и активности, а также внедрение машинного обучения для более точной персонализации тренировочных программ на основе долгосрочной статистики пользователя.

Таким образом, представленное решение не только решает актуальную проблему отсутствия гибкости и адаптивности в существующих фитнес-приложениях, но и создаёт основу для развития интеллектуальной системы персонального фитнес-сопровождения, способной учитывать индивидуальные особенности и текущее состояние каждого пользователя.
\newpage

\renewcommand{\refname}{Список используемых источников}
\bibliography{sources}
\newpage
\newpage

\section*{3. Выбор средств реализации}
\addcontentsline{toc}{section}{Приложения} % Добавляем в оглавление


\newpage

\end{document}
