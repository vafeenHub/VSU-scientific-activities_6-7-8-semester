
\subsection*{4.4. Проверка HTTP-запросов}
\addcontentsline{toc}{subsection}{4.4. Проверка HTTP-запросов} % Добавляем в оглавление

Ниже [на рисунках]
% , на рис. 
% ~\ref{fig:fig2}
% , рис. 
% ~\ref{fig:fig3}
% , рис. 
% ~\ref{fig:fig4}
% , рис. ~\ref{fig:fig5}
, изображены тестовые обращения для проверки запросов к собственному API через Postman. Все данные, необходимые для отображения пользователю, существуют и корректны.  \cite{ref9}



{\color{red} Здесь будут скриншоты из postman}
% \begin{figure}[H]
% 	\centering
% 	\includegraphics[width=0.8\textwidth]{images/2.png}
% 	\caption{Запрос получения преподавателей}
% 	\label{fig:fig2}
% \end{figure}

% \begin{figure}[H]
% 	\centering
% 	\includegraphics[width=0.8\textwidth]{images/3.png}
% 	\caption{Запрос получения групп}
% 	\label{fig:fig3}
% \end{figure}

% \begin{figure}[H]
% 	\centering
% 	\includegraphics[width=0.8\textwidth]{images/4.png}
% 	\caption{Запрос получения расписания конкретной группы}
% 	\label{fig:fig4}
% \end{figure}

% \begin{figure}[H]
% 	\centering
% 	\includegraphics[width=0.8\textwidth]{images/5.png}
% 	\caption{Запрос получения расписания конкретного преподавателя}
% 	\label{fig:fig5}
% \end{figure} 
