
\subsection*{3.3. Обзор существующих фреймворков базы данных}
\addcontentsline{toc}{subsection}{3.3. Обзор существующих фреймворков базы данных} % Добавляем в оглавление

При выборе подхода к работе с базами данных в Android-приложениях рассматривались два основных решения: нативный SQLite и библиотека Room. Нативный SQLite, являясь стандартным решением для Android, требует значительных усилий при реализации - разработчику необходимо вручную создавать таблицы, писать SQL-запросы и управлять соединениями, что приводит к большому объему шаблонного кода. В отличие от этого, Room предоставляет удобную абстракцию над SQLite, автоматически генерируя необходимый код на основе аннотаций, что значительно сокращает время разработки и уменьшает вероятность ошибок.
С точки зрения производительности оба подхода демонстрируют сопоставимые результаты, так как Room в конечном итоге использует тот же SQLite. Однако Room предлагает дополнительные оптимизации, такие как кэширование запросов и проверка их корректности на этапе компиляции, что позволяет избежать распространенных ошибок, характерных для ручного написания SQL-запросов. При этом нативный SQLite сохраняет преимущество при работе с особо сложными запросами, где может потребоваться тонкая ручная оптимизация.
Гибкость нативного SQLite проявляется в возможности выполнения произвольных запросов и полном контроле над структурой базы данных, однако это требует от разработчика глубоких знаний SQL и тщательного управления миграциями. Room, хотя и ограничивает некоторые низкоуровневые возможности, предоставляет удобные механизмы для работы с транзакциями и миграциями данных, существенно упрощая эти процессы. Особенно ценным является встроенная в Room поддержка LiveData и Flow, позволяющая легко реализовать реактивное обновление интерфейса при изменениях в базе данных.
Несмотря на то, что нативный SQLite имеет более обширную базу знаний и примеров благодаря своей долгой истории, Room активно развивается как часть Android Jetpack и становится стандартом для новых проектов. Его популярность постоянно растет, а сообщество разработчиков расширяется, что обеспечивает хорошую поддержку и наличие актуальных решений для типовых задач.
Важным преимуществом Room являются встроенные инструменты тестирования, такие как возможность создания временных баз данных в памяти, что существенно упрощает процесс написания и выполнения unit-тестов по сравнению с нативным SQLite, где требуется дополнительная настройка тестового окружения.
Таким образом, Room был выбран в качестве основного инструмента для работы с базами данных благодаря своей простоте использования, безопасности типов, отличной интеграции с другими компонентами Android Jetpack и возможности сосредоточиться на бизнес-логике приложения, а не на низкоуровневых деталях работы с SQLite. \cite{ref3}
