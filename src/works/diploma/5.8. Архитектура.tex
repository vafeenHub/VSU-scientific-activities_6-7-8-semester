
\subsection*{4.8. Архитектура приложения}
\addcontentsline{toc}{subsection}{4.8. Архитектура приложения} % Добавляем в оглавление

Приложение будет разработано с соблюдением принципов SOLID и чистой архитектуры. Применение этих принципов обеспечивает высокую степень модульности, гибкости и тестируемости кода, а также упрощает поддержку и расширение функциональности. \cite{ref3}

Чистая архитектура разделяет систему на несколько слоев, каждый из которых имеет чётко определённые задачи и зависимости. Это позволяет легко изменять и масштабировать приложение, а также улучшает его тестируемость. В соответствии с чистой архитектурой в приложении выделены следующие основные слои:

\begin{enumerate}
    \item \textbf{Внешний слой (UI Layer)}: включает все компоненты, связанные с пользовательским интерфейсом, такие как фрагменты и активности. Основная задача UI Layer - отображение данных и взаимодействие с пользователем. Этот слой получает данные из ViewModel и обновляет интерфейс в соответствии с изменениями в данных~\cite{ref3}.
    \item \textbf{Слой данных (Data Layer)}: содержит компоненты, связанные с управлением данными, включая DAO-интерфейсы для работы с Room, реализации сетевых запросов с использованием Retrofit, а также клиентов базы данных и сетевых клиентов. Data Layer отвечает за получение и хранение данных, предоставляя их в Domain Layer через репозитории~\cite{ref3}.
    \item \textbf{Слой домена (Domain Layer)}: содержит бизнес-логику и бизнес-модели. Domain Layer не зависит от других слоев и содержит основные бизнес-правила и интерфейсы репозиториев~\cite{ref3}.
    \item \textbf{Слой приложений (Application Layer)}: содержит ViewModel, которые объединяют бизнес-логику с логикой представления и управления состоянием. ViewModel взаимодействуют с репозиториями для получения данных и обработки пользовательских действий.
\end{enumerate}

Для разделения представления и логики управления данными в приложении используется паттерн MVVM (Model-View-ViewModel). Этот паттерн обеспечивает чёткое разделение обязанностей между компонентами приложения:

\begin{enumerate}
    \item \textbf{Model}: включает бизнес-логику и данные приложения. Модель получает данные из Data Layer и предоставляет их ViewModel.
    \item \textbf{View}: состоит из пользовательского интерфейса и отвечает за отображение данных. View наблюдает за изменениями в ViewModel и обновляет интерфейс в соответствии с этими изменениями.
    \item \textbf{ViewModel}: посредник между View и Model. ViewModel запрашивает данные у Model и предоставляет их View, а также обрабатывает пользовательские действия и обновляет Model.
\end{enumerate}

Для управления зависимостями и их внедрения в приложении используется библиотека Koin. В модулях Koin определяются зависимости и способы их создания. Например, можно определить зависимости для сетевого клиента, базы данных и ViewModel. Это позволяет централизованно управлять зависимостями и облегчает их модификацию. Koin автоматически предоставляет экземпляры классов, когда они требуются, что упрощает тестирование и модульность. Например, ViewModel могут получать необходимые репозитории через Koin, что устраняет жёсткие зависимости\cite{ref10}
