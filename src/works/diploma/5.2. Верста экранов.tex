\subsection*{4.2. Верстка экранов}
\addcontentsline{toc}{subsection}{4.2. Верстка экранов} % Добавляем в оглавление

% Все экраны приложения верстаются с помощью Kotlin Declarative подхода Jetpack Compose. На этом этапе вынесены отдельно строковые и цветовые ресурсы, необходимые шрифты и векторные изображения для иконок. Для всех частей интерфейса используются стандартные UI-элементы Android SDK.[2]
% Приложение состоит из двух основных экранов: расписание и настройки. Смена состояния основного Activity осуществляется при помощи помещения одного из двух фрагментов в Scaffold. Расписание занятий отображается с помощью Column, где каждый элемент списка – это контейнер для события, куда записывается информация о занятии. Пока не начата работа с сетью, оставлены изображения-заглушки и тестовый текст, чтобы убедиться в корректности отображения интерфейса. \cite{ref8}
% Для похожих элементов UI необходимо вынести одинаковые атрибуты в отдельный стиль для дальнейшего повторного использования. Также необходимо добавить поддержку светлой и темной темы, чтобы пользователи могли выбирать наиболее удобный для себя вариант отображения интерфейса. Ниже на рис. ~\ref{fig:main_screen} и рис. ~\ref{fig:settings_screen} изображена итоговая верстка экранов с тестовыми данными.

{\color{red}Здесь будет описание всех экранов с фотками верстки}
