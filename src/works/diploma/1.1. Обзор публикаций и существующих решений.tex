\subsection*{1.1. Обзор литературных источников и технологий}
\addcontentsline{toc}{subsection}{1.1. Обзор литературных источников и технологий}

Разработка современного фитнес-приложения требует опоры на актуальные технологические решения и понимание принципов организации тренировочного процесса. В рамках исследования были изучены как технические публикации, так и материалы, касающиеся методологии тренировок.

В области технологий стека Android ключевыми источниками выступила официальная документация Google для разработчиков \cite{ref5, ref6}. Особое внимание было уделено изучению возможностей фреймворка Jetpack Compose, который представляет собой парадигмальный сдвиг в создании пользовательских интерфейсов, предлагая декларативный подход вместо императивного \cite{ref5}. Анализ документации к библиотекам Room \cite{ref3} и Retrofit подтвердил их статус как стандартных решений для работы с локальным хранением данных и сетевыми запросами соответственно, обеспечивающих безопасность типов и сокращение шаблонного кода.

С методологической точки зрения, при проектировании логики приложения учитывались принципы прогрессивной перегрузки и необходимости адекватного восстановления \cite{ref2}. Идея адаптации тренировочного плана на основе обратной связи от пользователя, а не жесткого следования предустановленному графику, стала центральной. Исследования, касающиеся влияния качества и количества сна на когнитивные функции и физическое восстановление \cite{ref7}, обосновали необходимость включения в приложение не просто трекера сна, но и системы аналитики, выдающей содержательные рекомендации. Таким образом, техническая реализация опирается на проверенные отраслевые стандарты, а функциональное наполнение — на актуальные представления о эффективном и безопасном тренировочном процессе.