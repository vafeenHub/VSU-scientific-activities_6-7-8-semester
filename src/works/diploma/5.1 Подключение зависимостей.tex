\subsection*{4.1. Подключение зависимостей}
\addcontentsline{toc}{subsection}{4.1. Подключение зависимостей} % Добавляем в оглавление

Для написания приложения нужно добавить необходимые библиотеки в Gradle файле на уровне приложения: 
\begin{enumerate}
	\item Retrofit – это библиотека для работы с HTTP-запросами в Android-приложениях. Она используется для взаимодействия с собственным API, обеспечивая удобный и эффективный способ отправки запросов и обработки ответов. Конвертер Gson позволяет автоматически преобразовывать JSON-ответы от API в объекты Kotlin, что упрощает работу с данными.   \cite{ref7}\cite{ref8}
	
	\item Glide – это библиотека для загрузки и отображения изображений. Она используется для отображения картинок котиков на интерфейсе, а также их кэширования для повышения производительности. \cite{ref7}
	
	\item AndroidX Libraries предоставляет современные компоненты и инструменты для разработки Android-приложений, улучшая совместимость и функциональность. \cite{ref7}
	
	\item Jetpack Compose для декларативной верстки экранов приложения. \cite{ref7}
	
	\item Jetpack Navigation: обеспечивает удобный способ навигации между экранами приложения, упрощая управление компонентами и навигационными действиями. 
	
	\item Koin – это библиотека для внедрения зависимостей, которая упрощает создание и управление зависимостями в приложении, улучшая модульность и тестируемость кода.
	
	\item Room – это библиотека для работы с базой данных SQLite, предоставляющая удобный и безопасный способ хранения и управления данными в приложении. Благодаря ей будет происходить хранение информации о треках, плейлистах и избранном.
\end{enumerate}
