\section*{Заключение}
\addcontentsline{toc}{section}{Заключение} % Добавляем в оглавление

В ходе работы было разработано мобильное фитнес-приложение для платформы Android, реализующее комплексный подход к организации физических нагрузок и отслеживанию показателей здоровья. Приложение сочетает в себе основные преимущества существующих решений и дополняет их уникальным функционалом, основанным на принципах гибкости, адаптивности и персонализации.

Ключевыми достижениями работы являются:

\begin{itemize}
    \item Полноценный стек разработки: создание клиентской части с интуитивным интерфейсом и серверного приложения с безопасным API для авторизации и хранения данных.

    \item Локальная автономность и синхронизация: разработка модуля работы с данными, включающего локальную базу для стабильной работы оффлайн и надёжный механизм синхронизации с облачным сервером.

    \item Безопасность и надёжность: внедрение защищённого соединения через HTTPS с использованием SSL-сертификатов для обеспечения конфиденциальности пользовательских данных (логинов, прогресса, персональных планов).

    \item Комплексная бизнес-логика: реализация ядра приложения, включающего механизмы проведения тренировок, отслеживания прогресса и мониторинга показателей восстановления.

    \item Всестороннее тестирование: проведение комплексной проверки работоспособности, охватившей пользовательский интерфейс, функциональность клиентской и серверной частей, а также корректность ключевых алгоритмов.
\end{itemize}

Все поставленные задачи были успешно выполнены. Спроектирован и реализован пользовательский интерфейс с помощью Jetpack Compose, разработана локальная база данных на основе Room, создана бизнес-логика приложения, включая механизмы фильтрации тренировок и отслеживания прогресса, а также проведено тестирование работоспособности системы.

Разработанное приложение демонстрирует потенциал для дальнейшего масштабирования. В перспективе возможно добавление таких функций, как челленджи, интеграция с носимой электроникой для автоматического сбора данных о сне и активности, а также автоматическое определение начала тренировки по информации от носимого устройства. 

Таким образом, представленное решение не только решает актуальную проблему отсутствия гибкости и адаптивности в существующих фитнес-приложениях, но и создаёт основу для развития интеллектуальной системы персонального фитнес-сопровождения, способной учитывать индивидуальные особенности и текущее состояние каждого пользователя.
