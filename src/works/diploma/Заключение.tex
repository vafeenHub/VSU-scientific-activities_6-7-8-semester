\section*{Заключение}
\addcontentsline{toc}{section}{Заключение} % Добавляем в оглавление

В ходе работы было разработано мобильное фитнес-приложение для платформы Android, реализующее комплексный подход к организации физических нагрузок и отслеживанию показателей здоровья. Приложение сочетает в себе основные преимущества существующих решений и дополняет их уникальным функционалом, основанным на принципах гибкости, адаптивности и персонализации.

Ключевыми достижениями работы являются:
\begin{itemize}
    \item Интеграция готовой системы тренировок из интернет-ресурсов, где каждая сессия представляет собой самодостаточный комплекс, позволяющий пользователю гибко выбирать упражнения без риска нарушения общей программы;
    \item Реализация интеллектуального модуля адаптации сложности, анализирующего историю занятий и предлагающего корректировки нагрузки на основе устойчивости выполнения плана;
    \item Интеграция модуля учёта сна с системой аналитики и генерации персонализированных рекомендаций, что обеспечивает комплексный подход к восстановлению;
    \item Разработка системы «Советы по тренировкам», которая отслеживает процент завершённых упражнений и предлагает оптимизацию плана для предотвращения выгорания.
\end{itemize}

Все поставленные задачи были успешно выполнены: спроектирован и реализован пользовательский интерфейс с помощью Jetpack Compose, разработана локальная база данных на основе Room, реализована бизнес-логика приложения, включая механизмы фильтрации тренировок и отслеживания прогресса, а также проведено тестирование работоспособности системы.

Разработанное приложение демонстрирует потенциал для дальнейшего масштабирования. В перспективе возможно добавление социальных функций (соревнований, общих челленджей), интеграция с носимой электроникой для автоматического сбора данных о сне и активности, а также внедрение машинного обучения для более точной персонализации тренировочных программ на основе долгосрочной статистики пользователя.

Таким образом, представленное решение не только решает актуальную проблему отсутствия гибкости и адаптивности в существующих фитнес-приложениях, но и создаёт основу для развития интеллектуальной системы персонального фитнес-сопровождения, способной учитывать индивидуальные особенности и текущее состояние каждого пользователя.