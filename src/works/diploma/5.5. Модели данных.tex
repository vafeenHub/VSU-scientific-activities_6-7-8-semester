\subsection*{4.5. Модели данных}
\addcontentsline{toc}{subsection}{4.5. Модели данных} % Добавляем в оглавление

Следующие модели данных, изображенные на рис. {номера рисунков}
% ~\ref{fig:fig6}
, используются для конвертации в них HTTP-ответов от сервера в формат JSON и обработки для дальнейшего помещения в базу данных и отображения на интерфейсе. 

{\color{red}Здесь будут диаграммы моделей данных}
% \begin{figure}[H]
% 	\centering
% 	\includegraphics[width=0.7\textwidth]{images/6.png}
% 	\caption{Диаграмма классов для моделей данных сетевых репозиториев}
% 	\label{fig:fig6}
% \end{figure} 
% \begin{enumerate}
% 	\item LessonDto: модель для представления информации о занятии.
%  	\item TeacherDto: модель для представления информации о преподавателе.
% 	\item GroupDto: модель для представления информации о группе студентов.
% \end{enumerate}