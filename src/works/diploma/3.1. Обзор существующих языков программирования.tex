
\subsection*{3.1. Обзор существующих языков программирования}
\addcontentsline{toc}{subsection}{3.1. Обзор существующих языков программирования} % Добавляем в оглавление

При сравнении Java и Kotlin выявлены существенные различия в подходах к разработке. Java, будучи зрелым языком, требует больше ручного управления кодом, включая написание геттеров, сеттеров и других шаблонных конструкций, что увеличивает объем работы и вероятность ошибок. В то же время Kotlin предлагает более высокоуровневый подход с автоматической генерацией boilerplate-кода, например, в data-классах, что делает разработку быстрее и удобнее.
В плане производительности оба языка компилируются в байт-код JVM, однако Kotlin предоставляет встроенные оптимизации, такие как null-безопасность и корутины, которые упрощают асинхронное программирование и снижают риск ошибок. Java, хотя и остается высокопроизводительным, требует ручной оптимизации, особенно при работе с памятью и многопоточностью.
Гибкость Kotlin проявляется в поддержке функционального программирования, включая лямбда-выражения и расширения функций, что позволяет писать более лаконичный и модульный код. Java, несмотря на добавление Stream API и лямбд, все еще уступает в этом отношении, требуя большего количества шаблонного кода.
Поддержка и сообщество Java остаются одними из самых больших, что делает язык надежным выбором для legacy-проектов. Однако Kotlin активно развивается, особенно в Android-разработке, и полностью совместим с Java, что позволяет использовать существующие библиотеки и постепенно переходить на более современный стек.
Тестирование в Kotlin упрощается благодаря лаконичному синтаксису и встроенным возможностям, таким как корутины, тогда как в Java для сложных сценариев часто требуются дополнительные библиотеки. 
В итоге Kotlin был выбран как более современный, безопасный и удобный язык, сочетающий производительность Java с улучшенным синтаксисом и инструментами.
