\section*{2. Предлагаемое решение и его особенности}
\addcontentsline{toc}{section}{2. Предлагаемое решение и его особенности}

На основе выявленных недостатков существующих решений было сформулировано ядро предлагаемого приложения, которое позиционируется как интеллектуальный фитнес-компаньон. Его архитектура строится на трех фундаментальных принципах: гибкость, адаптивность и комплексность.

Во-первых, решена проблема жесткости тренировочных планов. В отличие от распространенной схемы, где пропуск одного занятия нарушает всю программу, каждая тренировка в приложении представляет собой самодостаточный комплекс. Пользователь имеет полный контроль над выполняемым набором упражнений внутри сессии, что позволяет гибко подстраивать нагрузку под текущее состояние, время и цели без чувства «сорыва плана».

Во-вторых, реализована система динамической адаптации сложности. Приложение анализирует историю тренировок пользователя. Алгоритм отслеживает как последовательные успешные завершения (предлагая увеличение нагрузки), так и рассогласование между средней и плановой продолжительностью занятий, рекомендуя в последнем случае скорректировать структуру тренировки для предотвращения выгорания. Эта функция закладывает основу для персонального подхода.

В-третьих, приложение выходит за рамки простого трекера упражнений, интегрируя модуль учета и анализа сна. Пользователь вручную вносит данные о продолжительности отдыха, а система, аккумулируя статистику, предоставляет не только графики, но и содержательные текстовые рекомендации, предупреждая о рисках, связанных с хроническим недосыпом или избыточным сном.

Реализованная функциональность включает:
\begin{itemize}
    \item бэкенд с системой регистрации и аутентификации пользователей;
    \item каталог готовых тренировок с возможностью гибкой настройки параметров выполнения;
    \item возможность добавления собственных планов тренировок;
    \item процесс проведения тренировки со встроенным таймером и фоновым аудиосопровождением;
    \item полную историю выполненных занятий;
    \item модуль для ручного ввода и статистического анализа продолжительности сна с генерацией советов;
    \item модуль «Советы по тренировкам», который в режиме реального времени анализирует историю активности пользователя: если система фиксирует несколько незавершенных тренировок подряд, она предлагает адаптировать план — например, уменьшить количество упражнений или их продолжительность, обеспечивая тем самым персонализированный и устойчивый прогресс.
\end{itemize}

Таким образом, разработанное приложение представляет собой целостную экосистему, которая не только предоставляет инструменты для тренировок, но и активно анализирует пользовательское поведение. За счет интеграции модулей адаптации сложности, анализа сна и интеллектуальных советов по тренировкам, система формирует замкнутый цикл обратной связи «приложение–пользователь». Это позволяет динамически корректировать нагрузку, предотвращая выгорание и способствуя достижению фитнес-целей с учетом индивидуальных особенностей и текущего состояния пользователя.
