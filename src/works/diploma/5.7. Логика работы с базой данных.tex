
\subsection*{4.7. Логика работы с базой данных}
\addcontentsline{toc}{subsection}{4.7. Логика работы с базой данных} % Добавляем в оглавление

Для обеспечения надежного и эффективного хранения данных о занятиях в приложении используется библиотека Room, которая предоставляет удобный интерфейс для работы с SQLite базой данных в Android-приложениях. Для реализации функциональности работы с базой данных в приложении используются следующие основные компоненты:

{\color{red}Здесь будет диаграмма классов работы с базой данных, диаграмма схемы базы данных }
% \begin{enumerate}
% 	\item Entity-классы представляют собой модели данных, которые будут сохраняться в базе данных. В приложении используются следующие Entity-классы, которые изображены ниже на рис. ~\ref{fig:fig8}:
% 		\begin{itemize}
% 		\item LessonEntity: представляет информацию о занятии, включая его идентификатор, название, время начала и конца, аудиторию, преподавателя и т. д.
% 		\item TeacherEntity: представляет информацию о преподавателе, включая его идентификатор, имя, должность и т. д.
% 		\item GroupEntity: представляет информацию о группе студентов, включая ее идентификатор, название и т. д.
% 		\item ReminderEntity: представляет информацию о напоминании о занятии
% 		\end{itemize}
% 		\begin{figure}[H]
% 			\centering
% 			\includegraphics[width=0.8\textwidth]{images/8.png}
% 			\caption{Схема базы данных}
% 			\label{fig:fig8}
% 		\end{figure} 
		
% 	\item DAO (Data Access Object) интерфейсы определяют методы для взаимодействия с базой данных. В нашем приложении используются три DAO-интерфейса, которые изображены на рис ~\ref{fig:fig9}:
% 		\begin{itemize}
% 		\item LessonDao: определяет методы для вставки, удаления и получения занятий из базы данных.
% 		\item TeacherDao: определяет методы для вставки, удаления и получения преподавателей из базы данных.
% 		\item GroupDao: определяет методы для вставки, удаления и получения групп из базы данных.
% 		\end{itemize}
% \end{enumerate}



% Эти интерфейсы аннотированы с использованием аннотаций Room, таких как \texttt{@Insert}, \texttt{@Delete} и \texttt{@Query}, которые определяют SQL-запросы, выполняемые для соответствующих операций. \cite{ref10}

% \begin{figure}[H]
% 	\centering
% 	\includegraphics[width=0.8\textwidth]{images/9.png}
% 	\caption{Диаграмма классов базы данных}
% 	\label{fig:fig9}
% \end{figure} 


% \begin{enumerate}[start=3]
%     \item Абстрактный класс базы данных наследуется от \texttt{RoomDatabase} и представляет собой точку входа для взаимодействия с базой данных. В нашем приложении определены следующие классы:
    
%     \begin{itemize}
%         \item AppDatabase: абстрактный класс для работы с таблицей занятий. Содержит абстрактные свойства \texttt{lessonDao}, \texttt{groupDao}, \texttt{teacherDao}, возвращающие экземпляры интерфейсов \texttt{LessonDao}, \texttt{GroupDao} и \texttt{TeacherDao} соответственно.
%     \end{itemize}

%     \item Клиент базы данных предоставляет методы для взаимодействия с базой данных через DAO-интерфейсы. В нашем приложении используется класс \texttt{AppDatabase\_Impl}, который реализует интерфейс \texttt{AppDatabase}. Этот класс определяет методы для сохранения, удаления и получения занятий. Класс \texttt{AppDatabase\_Impl} инкапсулирует логику работы с базой данных, обеспечивая удобный интерфейс для других компонентов приложения. Он использует инъекцию зависимости для получения экземпляра базы данных и обеспечивает выполнение CRUD-операций через соответствующие DAO-интерфейсы.
% \end{enumerate}