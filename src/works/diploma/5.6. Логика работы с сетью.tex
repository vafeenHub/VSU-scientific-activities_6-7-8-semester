\subsection*{4.6. Логика работы с сетью}
\addcontentsline{toc}{subsection}{4.6. Логика работы с сетью} % Добавляем в оглавление

Основная цель работы с сетью — обеспечение взаимодействия приложения с API, для получения информации о расписании по запросу пользователя. \cite{ref6}

Для работы с сетью в приложении используются следующие ключевые компоненты, диаграмма которых изображена на рис. 
% \ref{fig:fig7}
: 

{\color{red}Здесь будет диаграмма репозиториев для работы с сетью}


% \begin{enumerate}
%     \item Интерфейсы LessonRemoteRepository, GroupRemoteRepository, Teacher-RemoteRepository. Определяют методы, которые принимает объект запроса и возвращает объект ответа. Эти интерфейсы позволяет абстрагироваться от конкретных деталей реализации сетевого взаимодействия, обеспечивая гибкость и возможность замены реализации при необходимости.
	
%     \item Классы LessonRemoteRepositoryImpl, GroupRemoteRepositoryImpl, TeacherRemoteRepositoryImpl. Реализуют интерфейсы Lesson-RemoteRepository, GroupRemoteRepository, TeacherRemoteRepository и используют библиотеку Retrofit для выполнения сетевых запросов. Здесь проверяется сетевое подключение – прежде чем отправить запрос, проверяется наличие активного интернет-соединения. Если соединение отсутствует, возвращается ответ с соответствующим кодом ошибки. А также обрабатывается сам запрос: методы принимают параметры запроса и выполняют сетевой запрос к API. В случае успешного выполнения запроса возвращается ответ с кодом 200 и полученными данными. В случае ошибки (например, при отсутствии соединения или неверном типе запроса) возвращаются соответствующие коды ошибок.\cite{ref9}
% \end{enumerate}

% \begin{figure}[H]
% 	\centering
% 	\includegraphics[width=0.8\textwidth]{images/7.png}
% 	\caption{Диаграмма классов для сетевых репозиториев}
% 	\label{fig:fig7}
% \end{figure}