\section*{2. Обзор литературных источников и технологий}
\addcontentsline{toc}{section}{2. Обзор литературных источников и технологий}

Разработка современного фитнес-приложения, претендующего на роль интеллектуального помощника, требует комплексного анализа двух основных областей: актуальных технологических стеков для мобильной разработки и научно обоснованных принципов построения тренировочного процесса и восстановления. В рамках данного исследования был проведён детальный обзор как технических публикаций и платформенной документации, так и современных методических и научных работ в области физиологии спорта и нейронаук.

\subsection*{2.1. Технологический стек и архитектурные подходы}
\addcontentsline{toc}{subsection}{2.1. Технологический стек и архитектурные подходы}

Основу технической реализации проекта составило изучение официальной документации и рекомендаций для разработчиков на платформе Android \cite{ref5, ref6}. Современная экосистема Android сместилась в сторону декларативных UI-фреймворков, что подтверждается глубоким анализом возможностей Jetpack Compose \cite{ref5}. Этот инструмент не просто упрощает создание интерфейсов, но и способствует построению более отзывчивых и легких в поддержке приложений за счёт реактивной модели данных и композиции компонентов. Изучение архитектурных шаблонов, сопутствующих Compose (таких как Model-View-ViewModel или MVI), выявило их преимущества в управлении состоянием UI и отделении бизнес-логики от кода отображения, что критически важно для долгосрочной поддержки проекта.

Для организации работы с данными были рассмотрены стандартные решения. Библиотека Room \cite{ref3}, как часть Android Jetpack, подтвердила свою эффективность в качестве абстракции над SQLite, предоставляя безопасность типов на уровне компиляции, удобные аннотации и поддержку наблюдения за изменениями данных через интеграцию с LiveData или Flow. Сетевой слой требовал надёжного решения для REST API, и выбор пал на Retrofit, который де-факто является отраслевым стандартом благодаря генерации реализации интерфейсов, встроенной поддержке различных конвертеров (включая Gson и Moshi) и высокой степени кастомизации через интерцепторы. Особое внимание было уделено изучению паттернов объединения локального кэша (Room) и сетевых данных, включая стратегии синхронизации и обработки офлайн-сценариев.

Также в обзор попали технологии для асинхронного программирования, где Kotlin Coroutines в сочетании с Flow продемонстрировали значительные преимущества перед традиционными callback-подходами и RxJava в контексте Kotlin-first разработки, предлагая более простой и читаемый код для выполнения фоновых операций.

\subsection*{2.2. Методологические основы тренировок и восстановления}
\addcontentsline{toc}{subsection}{2.2. Методологические основы тренировок и восстановления}

Функциональное наполнение приложения проектировалось с опорой на современные спортивные методологии. Ключевым стал анализ принципа прогрессивной перегрузки, согласно которому для роста физических показателей необходимо систематически увеличивать нагрузку на организм \cite{ref2}. Однако исследования подчёркивают, что этот рост нелинеен и зависит от индивидуальных адаптационных возможностей и качества восстановления. Это обосновало отказ от жёстких, универсальных тренировочных планов в пользу системы, которая адаптирует рекомендации по тренировочной нагрузке на основе истории выполнения пользователем упражнений.

Анализ научных работ, посвящённых влиянию сна на физическую и когнитивную производительность \cite{ref7}, показал, что недостаток или низкое качество сна являются критическими факторами, ограничивающими прогресс. Они приводят к снижению эффективности восстановления, ухудшению моторного контроля и повышению риска травм. Эти данные сформировали методологическую основу для интеграции ручного трекера восстановления, где сон рассматривается как активный компонент тренировочного процесса, равный по значимости нагрузке.

Таким образом, методология приложения основывается на двух ключевых переменных: тренировочной нагрузке (выполненный объём упражнений) и продолжительности сна. Алгоритм анализирует эти данные по простому принципу, формируя основу для адаптивного планирования \cite{ref10}.

\subsection*{2.3. Анализ существующих решений и рыночных тенденций}
\addcontentsline{toc}{subsection}{2.3. Анализ существующих решений и рыночных тенденций}

Для определения уникального ценностного предложения был проведён сравнительный анализ популярных фитнес-приложений. Изучались специализированные программы для силового тренинга (Strong, Hevy, Jefit), комплексные фитнес-трекеры (MyFitnessPal, Adidas Training) и встроенные решения от производителей умных часов (Samsung Health, Google Fit). Выявились общие тенденции, определяющие текущие рыночные пробелы.

Большинство приложений сосредоточено на одной из двух функций: либо это цифровой дневник для регистрации подходов и повторений с обширной, но статичной библиотекой упражнений, либо это агрегатор данных из различных источников (шаги, пульс, сон), слабо связанных с конкретными тренировочными целями. Планы тренировок часто представляют собой жесткие, неизменяемые шаблоны, рассчитанные на «среднего» пользователя. В них отсутствует адаптивность, то есть способность автоматически подстраивать сложность и объём нагрузки в ответ на фактические результаты и состояние пользователя. Ключевой недостаток — это разрыв между данными о нагрузке (тренировки) и восстановлении (сон, утомление). Даже при наличии трекеров сна эта информация редко используется для модификации последующих тренировочных заданий.

Таким образом, рынок заполнен приложениями, которые действуют либо как пассивный журнал, либо как сборщик разрозненных метрик, но не как активный, интеллектуальный помощник, принимающий решения. Однако научные обзоры и маркетинговые исследования указывают на растущий спрос со стороны пользователей на более персонализированные и «умные» подходы в сфере здоровья и фитнеса. Пользователи ожидают, что приложение не просто хранит данные, но анализирует их и даёт понятные, обоснованные рекомендации, которые эволюционируют вместе с их прогрессом и текущим состоянием. Это создаёт запрос на системы, способные на простой, но эффективный анализ введённой информации для персонального руководства, что и является основным направлением для разрабатываемого решения.

\subsection*{2.4. Синтез требований к разрабатываемому приложению}
\addcontentsline{toc}{subsection}{2.4. Синтез требований к разрабатываемому приложению}

Синтез полученных из литературных источников знаний позволил сформулировать ключевые требования к системе:

\begin{enumerate}
    \item \textbf{Технологические:} использование современных, поддерживаемых инструментов (Jetpack Compose, Coroutines, Room, Retrofit) для создания стабильного, производительного и легко развиваемого продукта.
    \item \textbf{Архитектурные:} реализация чистого кода с чётким разделением ответственности, что обеспечит гибкость при внедрении новых функций (например, интеграции с новыми сервисами).
    \item \textbf{Методологические:} воплощение принципов адаптивного планирования, при котором система анализирует выполнение плана тренировок и данные о сне для персонализированной коррекции нагрузки.
\end{enumerate}

Таким образом, проведённый обзор подтвердил, что эффективное фитнес-приложение нового поколения должно представлять собой симбиоз технической реализации, построенной на отраслевых стандартах, и интеллектуальной, научно обоснованной логики, которая персонифицирует тренировочный процесс и делает акцент на целостном управлении здоровьем, где сон и восстановление являются не менее важными переменными, чем сама нагрузка.
