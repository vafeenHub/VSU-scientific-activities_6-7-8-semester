\section*{2. Обзор литературных источников и технологий}
\addcontentsline{toc}{section}{2. Обзор литературных источников и технологий}

Разработка современного фитнес-приложения, претендующего на роль интеллектуального помощника, требует комплексного анализа двух основных областей: актуальных технологических стеков для мобильной разработки и научно обоснованных принципов построения тренировочного процесса и восстановления. В рамках данного исследования был проведён детальный обзор как технических публикаций и платформенной документации, так и современных методических и научных работ в области физиологии спорта и нейронаук.

\subsection*{2.1. Технологический стек и архитектурные подходы}
\addcontentsline{toc}{subsection}{2.1. Технологический стек и архитектурные подходы}

Основу технической реализации проекта составило изучение официальной документации и рекомендаций для разработчиков на платформе Android \cite{ref5, ref6}. Современная экосистема Android сместилась в сторону декларативных UI-фреймворков, что подтверждается глубоким анализом возможностей Jetpack Compose \cite{ref5}. Этот инструмент не просто упрощает создание интерфейсов, но и способствует построению более отзывчивых и легких в поддержке приложений за счёт реактивной модели данных и композиции компонентов. Изучение архитектурных шаблонов, сопутствующих Compose (таких как Model-View-ViewModel или MVI), выявило их преимущества в управлении состоянием UI и отделении бизнес-логики от кода отображения, что критически важно для долгосрочной поддержки проекта.

Для организации работы с данными были рассмотрены стандартные решения. Библиотека Room \cite{ref3}, как часть Android Jetpack, подтвердила свою эффективность в качестве абстракции над SQLite, предоставляя безопасность типов на уровне компиляции, удобные аннотации и поддержку наблюдения за изменениями данных через интеграцию с LiveData или Flow. Сетевой слой требовал надёжного решения для REST API, и выбор пал на Retrofit, который де-факто является отраслевым стандартом благодаря генерации реализации интерфейсов, встроенной поддержке различных конвертеров (включая Gson и Moshi) и высокой степени кастомизации через интерцепторы. Особое внимание было уделено изучению паттернов объединения локального кэша (Room) и сетевых данных, включая стратегии синхронизации и обработки офлайн-сценариев.

Также в обзор попали технологии для асинхронного программирования, где Kotlin Coroutines в сочетании с Flow продемонстрировали значительные преимущества перед традиционными callback-подходами и RxJava в контексте Kotlin-first разработки, предлагая более простой и читаемый код для выполнения фоновых операций.

\subsection*{2.2. Методологические основы тренировок и восстановления}
\addcontentsline{toc}{subsection}{2.2. Методологические основы тренировок и восстановления}

Функциональное наполнение приложения проектировалось с опорой на современные спортивные методологии и данные физиологических исследований. Ключевым стал анализ принципа прогрессивной перегрузки, согласно которому для роста физических показателей необходимо систематически увеличивать нагрузку на организм \cite{ref2}. Однако в исследованиях подчёркивается, что этот рост нелинеен и зависит от индивидуальных адаптационных возможностей. Это обосновало отказ от жёстких, универсальных тренировочных планов в пользу адаптивной системы, которая модифицирует интенсивность, объём и характер упражнений на основе обратной связи от пользователя (субъективная оценка сложности, реальные результаты выполнения, темп прогресса).

Вторая фундаментальная ось — процесс восстановления. Был изучен комплекс факторов, включающий не только пассивный отдых, но и питание, ментальный стресс и, что наиболее важно, сон. Анализ работ, посвящённых влиянию сна на физическую и когнитивную производительность \cite{ref7}, показал, что недостаток или низкое качество сна приводят к снижению синтеза белка, нарушению регуляции гормонов (кортизол, тестостерон, гормон роста), ухудшению моторного контроля и повышению риска травм. Эти данные перевели трекер сна из разряда вспомогательных функций в ключевой диагностический модуль. Приложение должно не только регистрировать длительность сна, но и анализировать его структуру (по возможности интеграции с внешними датчиками), оценивать консистенцию режима и на основе этих данных выдавать персонализированные рекомендации: от корректировки времени тренировки до советов по гигиене сна.

\subsection*{2.3. Анализ существующих решений и рыночных тенденций}
\addcontentsline{toc}{subsection}{2.3. Анализ существующих решений и рыночных тенденций}

Для определения уникального ценностного предложения был проведён сравнительный анализ популярных фитнес-приложений (таких как Strong, Hevy, Jefit, а также встроенных решений от производителей умных часов). Выявились общие тенденции: перегруженность шаблонными планами, слабая связь между данными о тренировках, питании и сне, отсутствие выраженной адаптивности. Многие приложения действуют по принципу цифрового дневника, а не интеллектуального тренера. Научные обзоры указывают на растущий спрос на персонализированные, data-driven подходы в wellness-индустрии, где решения основываются на непрерывном мониторинге и интерпретации множества физиологических сигналов.

\subsection*{2.4. Синтез требований к разрабатываемому приложению}
\addcontentsline{toc}{subsection}{2.4. Синтез требований к разрабатываемому приложению}

Синтез полученных из литературных источников знаний позволил сформулировать ключевые требования к системе:

\begin{enumerate}
    \item \textbf{Технологические:} использование современных, поддерживаемых инструментов (Jetpack Compose, Kotlin Coroutines, Room, Retrofit) для создания стабильного, производительного и легко развиваемого продукта.
    \item \textbf{Архитектурные:} реализация чистого кода с чётким разделением ответственности, что обеспечит гибкость при внедрении новых функций (например, интеграции с новыми типами датчиков или сервисами).
    \item \textbf{Методологические:} воплощение принципов адаптивного планирования тренировок, где алгоритм учитывает историю нагрузок, скорость восстановления и субъективное состояние пользователя.
    \item \textbf{Аналитические:} предоставление пользователю не сырых данных, а осмысленной аналитики и actionable insights, особенно в области взаимосвязи сна, восстановления и эффективности тренировок.
\end{enumerate}

Таким образом, проведённый обзор подтвердил, что эффективное фитнес-приложение нового поколения должно представлять собой симбиоз robust-технической реализации, построенной на отраслевых стандартах, и интеллектуальной, научно обоснованной логики, которая персонифицирует тренировочный процесс и делает акцент на целостном управлении здоровьем, где сон и восстановление являются не менее важными переменными, чем сама нагрузка.