\section*{5. Выбор средств реализации}
\addcontentsline{toc}{section}{5. Выбор средств реализации} % Добавляем в оглавление

В качестве языка программирования был выбран Kotlin, потому что это современный, лаконичный и мощный язык, который устраняет многие недостатки Java. Kotlin предлагает такие функции, как null-безопасность, корутины, data-классы и расширения, которые значительно упрощают разработку и делают код более читаемым и поддерживаемым. Кроме того, Kotlin полностью совместим с Java, что позволяет использовать существующие библиотеки и постепенно мигрировать на новый язык. Этот выбор позволяет мне создавать более эффективные, безопасные и современные приложения с минимальными усилиями. В качестве среды разработки используется Android Studio. Она предоставляет все необходимые инструменты для создания, отладки и развертывания приложений для платформы Android. В ней можно создавать макеты пользовательского интерфейса, писать код на Kotlin, проводить отладку и тестирование приложения.

Внутри Android Studio используется Gradle – это система управления зависимостями и сборкой проектов, используемая в Android Studio. С помощью Gradle будут определяться зависимости и настройки проекта. \cite{ref4}

Для удобства разработки используется система контроля версий Git, которая используется для отслеживания изменений в файловой системе проекта. Она позволяет сохранять историю изменений, возвращаться к предыдущим версиям проекта, совместно работать над кодом с другими участниками и управлять кодом в разных ветках разработки. Git работает на уровне файловой системы и фиксирует изменения в файлах и папках, создавая так называемые "коммиты". Каждый коммит содержит информацию о том, какие файлы были изменены, кто и когда внес изменения, а также комментарии к изменениям. \cite{ref4}

Для написания интерфейса был выбран Jetpack Compose, так как это современный и мощный инструмент для создания пользовательских интерфейсов в Android. Compose позволяет описывать UI прямо в коде на Kotlin, что делает разработку более интуитивной и эффективной. Благодаря своей декларативной природе, Compose упрощает создание динамических и отзывчивых интерфейсов, а также поддерживает современные функции, такие как анимации и Material Design 3. Этот выбор позволяет мне создавать красивые и производительные приложения с минимальным количеством шаблонного кода. \cite{ref5}\cite{ref6}

Для работы с базой данных был выбран Room, потому что это высокоуровневая библиотека, которая значительно упрощает взаимодействие с SQLite. Room автоматизирует многие процессы, такие как создание таблиц, выполнение запросов и миграции, что позволяет сосредоточиться на логике приложения, а не на низкоуровневых деталях работы с базой данных. Благодаря интеграции с LiveData и Flow, Room также упрощает работу с данными в реальном времени, что делает его идеальным выбором для современных Android-приложений. 

Для работы с сетевыми запросами был выбран Retrofit, так как это одна из самых популярных и удобных библиотек для взаимодействия с REST API. Retrofit позволяет описывать API с помощью интерфейсов и аннотаций, что делает код чистым и легко читаемым. Благодаря своей интеграции с OkHttp, Retrofit обеспечивает высокую производительность и гибкость, а также поддерживает множество функций, таких как автоматическая сериализация/десериализация данных и обработка ошибок. Этот выбор позволяет мне эффективно работать с сетевыми запросами и сосредоточиться на бизнес-логике приложения. 

Для тестирования сервера используется online-инструмент Postman, который позволяет отправлять HTTP-запросы к серверу и проверять ответы. Он позволяет создавать и отправлять запросы различных типов (GET, POST, PUT, DELETE и т. д.), настраивать заголовки и параметры запроса, а также автоматизировать тестирование с помощью коллекций запросов и сценариев. Это будет полезно при работе с внешним API для интеграции с расписанием.\cite{ref9}
