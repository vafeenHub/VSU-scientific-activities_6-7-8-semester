
\subsection*{4.3. Реализация навигации}
\addcontentsline{toc}{subsection}{4.3. Реализация навигации} % Добавляем в оглавление
На данном этапе необходимо связать сверстанные экраны в единое приложение. Для реализации логики Bottom Bar будет использоваться Jetpack Navigation. При нажатии на элементы расписания необходимо обеспечить переход на соответствующие детальные экраны или выполнить другие действия, такие как открытие дополнительной информации о занятии. \cite{ref8}

{\color{red} Здесь будут скрины экранов}
% \begin{figure}[H]
%     \centering
%     \begin{minipage}{0.48\textwidth}
%         \centering
%         \includegraphics[width=\linewidth]{images/13.png}
%         \caption{Макет главного экрана}
%         \label{fig:main_screen}
%     \end{minipage}
%     \hfill
%     \begin{minipage}{0.48\textwidth}
%         \centering
%         \includegraphics[width=\linewidth]{images/14.png}
%         \caption{Макет экрана настроек}
%         \label{fig:settings_screen}
%     \end{minipage}
% \end{figure}

% На экране настроек при нажатии кнопки «Назад» необходимо вернуться на предыдущий экран. Это реализовано с помощью стандартных механизмов навигации в Android, например, NavController. \cite{ref8}
% До реализации взаимодействия с базой данных полная функциональность некоторых экранов будет ограничена. Однако, основная логика навигации и переходов между экранами будет реализована для обеспечения плавного пользовательского опыта.
